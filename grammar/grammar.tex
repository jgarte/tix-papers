\subsection{Nix grammar}
\label{sec:nix-grammar}

\todo{Write the nix grammar}

\subsection{nix-light}
\label{sec:nix-light-grammar}

The grammar of nix-light is based of the grammar of nix and brings several
modifications:
\begin{itemize}
  \item The main difference concerns records, whose representation is simplified as
    much as possible:

    \begin{itemize}
      \item Litteral records are always atomic (\emph{ie} only have one field). An
        \emph{orthogonal merge} operator ($\orthplus$) allows building complex
        records~\footnote{We still often write \{ $s_1$ = $y_1$; $\cdots{}$;
        $s_n$ = $y_n$ \} as a shorthand for \{ $s_1$ = $y_1$; \} $\orthplus
        \cdots{} \orthplus$ \{ $s_n$ = $y_n$; \}},

      \item The syntax in nix for a recursive definition of records (\texttt{\{ x.y
        = 1; x.z = 2; \}}) doesn't exist anymore,

      \item The labels can only be nix expressions (so litteral labels must be
        written as constant strings).
    \end{itemize}

  \item Another huge change is the removal of the \emph{if} construct which is
    replaced by a more general \emph{typecase} which is easier to reason on.

  \item The third notable change is that the opaque list construct of nix is
    replaced by the classical \texttt{nil} and \texttt{cons}.
    This avoids having over-complicated typing and evaluation rules for lists.

    For consistency, a pattern for lists has also been added.
\end{itemize}

The grammar of nix-light is given in the
figures~\pref{grammar::expressions},~\pref{grammar::values}
and~\pref{grammar::types}.

\begin{figure}
  \begin{grammar}
  \bfseries
  <e> ::=
    $x$ \| $c$
    \alt $e$.$a$ \| $e$.$a$ or $e$
    \alt $\λ$$p$.$e$ \| $e$ $e$
    \alt \{ $e$ = $e$ \}
    \alt with $e$; $e$
    \alt ($x$ := $e$ $\bm{\in}$ $t$) ? $e$ : $e$
    \alt let $x$ = $e$; $\cdots{}$; $x$ = $e$; in $e$
    \alt let rec $r$ = $e$; $\cdots{}$; $r$ = $e$; in $e$
    \alt $o$

    <o> ::= attrNames($e$) \| listToAttrs($e$)
    \alt $e$ + $e$ \| $e$ $\orthplus$ $e$ \| $e$ \textbackslash $e$
    \alt deepSeq($e$, $e$)
    \alt functionArgs($e$)
    % Maybe this can be typed as a function, but it is rather unlikely
    \alt Cons($e$, $e$)

  <a> ::= $e$. $\cdots{}$ .$e$

  <c> ::= \meta{s} \| \meta{i} \| $\cdots{}$
    \alt nil
    \alt $\cdots{}$

  <p> ::= $q$ \| $q$@$x$ \| $r$

  <q> ::= \{ $f$, $\cdots{}$, $f$ \} \| \{ $f$, $\cdots{}$, $f$ , \ldots{}\}
    \alt Cons($r$, $r$) \| nil
    \alt $q$:\τ

  <r> ::= $x$ | $x$:\τ

  <f> ::= $r$ \| $r$ ? $c$

\end{grammar}

  \caption{\label{grammar::expressions}The nix-light grammar for expressions}
\end{figure}

\begin{figure}
  \begin{grammar}
  \bfseries

  <v> ::=
    \c/
    \alt $\λ$\p/.\e/
    \alt \{ \v/ = \e/ \} $\orthplus \cdots{} \orthplus$ \{ \v/ = \e/ \}
    \alt Cons(\e/, \e/) \| nil
\end{grammar}

  \caption{\label{grammar::values}The nix-light grammar for values}
\end{figure}

\begin{figure}
  \begin{grammar}
  \bfseries
  <t> ::= \c/ \| \t/ $\bm{\rightarrow}$ \t/
    \alt \t/ $\bm{\vee}$ \t/ \| \t/ $\bm{\wedge}$ \t/ \| \t/ $\bm{\backslash}$ \t/
    \alt [\meta{R}]
    \alt \{ \s/ = \u/; $\cdots{}$; \s/ = \u/; _ = \u/ \}
    \alt bool \| int \| string

  <u> ::= \t/ \| ?\t/

  <R> ::= \t/ \| \meta{R^{\bm{+}}} \| \meta{R}* \| \meta{R}?
    \| \meta{R} \meta{R} \| \meta{R}\texttt{|}\meta{R}

  <\τ> ::= \t/ % No polymorphism for now
\end{grammar}

We write \textbf{\{ \s/ = \u/; $\cdots{}$; \s/ = \u/; \}} as
syntactic sugar for \textbf{\{ \s/ = \u/; $\cdots{}$; \s/ = \u/; _ =
$\undef$ \}}, and \textbf{\{ \s/ = \u/; $\cdots{}$; \s/ = \u/;.. \}}
as syntactic sugar for \textbf{\{ \s/ = \u/; $\cdots{}$; \s/ = \u/; _
= \textmd{\emph{Any}} \}}

  \caption{\label{grammar::types}The nix-light grammar for types}
\end{figure}
