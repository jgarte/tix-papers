\documentclass{article}

\usepackage[margin=1in]{geometry}
\usepackage{fontspec}
\usepackage[usenames,dvipsnames,svgnames,table]{xcolor}
\usepackage{amssymb}
\usepackage{amsmath}
\usepackage{bm}
\usepackage{stmaryrd}

\usepackage{mathtools}
\usepackage{mathpartir}
\newcommand{\irlabel}[1]{\text{\emph{(#1)}}}

\usepackage{xfrac}

\usepackage{listings}
\lstset{%
  escapeinside={//*}{*//}
  }

\usepackage{hyperref}
\usepackage{todo}

\date{}

\usepackage{syntax}
\renewcommand{\grammarlabel}[2]{\meta{#1 #2}}
\newcommand{\meta}[1]{\ensuremath{#1}} % For meta syntax
\renewcommand{\|}{\textrm{|}}
\def\a/{\meta{a}}
\def\b/{\meta{b}}
\def\c/{\meta{c}}
\def\e/{\meta{e}}
\def\E/{\meta{E}}
\def\f/{\meta{f}}
\def\o/{\meta{o}}
\def\p/{\meta{p}}
\def\q/{\meta{q}}
\def\s/{\meta{s}}
\def\t/{\meta{t}}
\def\u/{\meta{u}}
\def\v/{\meta{v}}
\def\x/{\meta{x}}

\newcommand{\assign}[2]{\ensuremath{\sfrac{#2}{#1}}}
\newcommand{\assignp} [2] {\assign{#1}{#2}}
\newcommand{\subst} [3] {#3 [\assign{#1}{#2}]}
\newcommand{\substp} [3] {#3 [\assignp{#1}{#2}]}
\newcommand{\dstep} [2] {#1 \ensuremath{\rightarrow} #2}
\newcommand{\ndstep} [2] {#1 \ensuremath{\nrightarrow} #2}
\newcommand{\ndsteps} [2] {#1 \ensuremath{\nrightarrow^*} #2}
\newcommand{\dstepa} [3] {\dstep{#1}{&#2}~\emph{#3} \\}

\newcommand{\eqdef}[2]{#1 \ensuremath{\overset{\text{def}}{=}} #2}
\newcommand{\eqdefa}[3]{\eqdef{#1}{&#2} \emph{#3} \\}

\newcommand{\xone}{\ensuremath{x_1}}
\newcommand{\xn}{\ensuremath{x_n}}
\newcommand{\eone}{\ensuremath{e_1}}
\newcommand{\etwo}{\ensuremath{e_2}}
\newcommand{\en}{\ensuremath{e_n}}
% TODO: redefine to get the 0 and 1 of types
% (with double bar)
\newcommand{\zero}{0}
\newcommand{\one}{1}


\newcommand{\ty}[1]{\texttt{#1}}
\newcommand{\set}[1]{\ensuremath{\mathcal{#1}}}
\newcommand{\undef}{\nabla}
\newcommand{\quasiconst}[1]{\overset{#1}{\twoheadrightarrow}}
\DeclareMathOperator\dom{dom}
\DeclareMathOperator\deff{def}
\newcommand{\orthsum}{\oplus^\bot}
\newcommand{\orthplus}{\diamond}
\newcommand{\subtype}{\leq}
\newcommand{\onerec}{\{ \textbf{..} \}}
\DeclareCollectionInstance{plainmath}{xfrac}{mathdefault}{math}
{%
  slash-symbol = \sslash{}
}
\newcommand{\ofTypeP}[2]{\UseCollection{xfrac}{plainmath}\sfrac{#1}{#2}}
\newcommand{\matchType}[1]{\Lbag #1 \Rbag}

\newcommand{\Γ}{\Gamma}
\newcommand{\τ}{\ensuremath{\tau}}
\newcommand{\σ}{\sigma}
\DeclareMathOperator\any{\textsc{Any}}
\newcommand{\λ}{\lambda}
\newcommand{\recleq}{\sqsubsetleq}
\newcommand{\discrete}[2]{\left\{ #1, .., #2 \right\}}

\newcommand{\pref}[1]{\ref{#1} at page~\pageref{#1}}


\title{Nix simplified grammar}
\begin{document}

\maketitle{}

\section{expressions}

\begin{grammar}
  \bfseries
  <e> ::=
    \x/ \| \c/
    \alt \e/.\a/ \| \e/.\a/ or \e/
    \alt \p/:\e/ \| \e/ \e/
    \alt \{ \e/ = \e/; \}
    \alt with \e/; \e/
    \alt (\e/ $\bm{\in}$ \t/) ? \e/ : \e/
    \alt let \x/ = \e/; $\cdots{}$; \x/ = \e/; in \e/
    \alt \o/

    <o> ::= attrNames(\e/) \| listToAttrs(\e/)
    \alt \e/ + \e/ \| \e/ $\orthplus$ \e/ \| \e/ \textbackslash \e/
    \alt deepSeq(\e/, \e/) \| seq(\e/, \e/)
    \alt functionArgs(\e/)
    % Maybe this can be typed as a function, but it is rather unlikely
    \alt Cons(\e/, \e/)

  <a> ::= \b/. $\cdots{}$ .\b/

  <b> ::= \$\{\e/\} \| \meta{l}

  <c> ::= \meta{s} \| \meta{i} \| $\cdots{}$
    \alt nil
    \alt $\cdots{}$

  <p> ::= \q/ \| \q/@\x/ \| \x/

  <q> ::= \{ \f/, $\cdots{}$, \f/ \} \| \{ \f/, $\cdots{}$, \f/ , \ldots{}\}
    \alt Cons(\x/, \x/) \| nil
    \alt \c/

  <f> ::= \x/ \| \x/ ? \e/

\end{grammar}

The grammar for types (\t/) is given in the \texttt{type-system} document.


\section{values}

\begin{grammar}
  \bfseries

  <v> ::=
    \c/
    \alt \p/:\e/
    \alt \{ \v/ = \e/; \} $\orthplus \cdots{} \orthplus$ \{ \v/ = \e/; \}
    \alt Cons(\e/, \e/) \| nil
\end{grammar}
\end{document}
