\documentclass{article}

\usepackage{fontspec}
\usepackage[usenames,dvipsnames,svgnames,table]{xcolor}
\usepackage{amssymb}
\usepackage{amsmath}
\usepackage{stmaryrd}
\usepackage{listings}
\lstset{%
  escapeinside={(*}{*)}
  }

\usepackage{mathtools}
\usepackage{mathpartir}
\usepackage{xfrac}
\DeclareCollectionInstance{plainmath}{xfrac}{mathdefault}{math}
{
  slash-symbol = \sslash
}
\newcommand{\ofTypeP}[2]{\UseCollection{xfrac}{plainmath}\sfrac{#1}{#2}}

\newcommand{\ty}[1]{\texttt{#1}}
\newcommand{\set}[1]{\ensuremath{\mathcal{#1}}}
\newcommand{\undef}{\oslash}
\newcommand{\quasiconst}[1]{\overset{#1}{\twoheadrightarrow}}
\DeclareMathOperator\dom{dom}
\newcommand{\orthsum}{\oplus^\bot}
\newcommand{\subtype}{\tilde{\leq}}

\newcommand{\Γ}{\Gamma}
\newcommand{\τ}{\tau}
\newcommand{\σ}{\sigma}
\newcommand{\recleq}{\sqsubsetleq}
\newcommand{\discrete}[2]{\left\{ #1, \ldots{}, #2 \right\}}

% TODO: redefine to get the 0 and 1 of types
% (with double bar)
\newcommand{\zero}{0}
\newcommand{\one}{1}

\title{A type system for nix}
\author{Théophane Hufschmitt}
\date{}

\begin{document}

\maketitle{}

\begin{mathpar}
  \inferrule{ }{\Gamma; x:\tau \vdash x:\tau}(Var)

  \and
  \inferrule{ }{\Gamma \vdash c:\mathcal{B}(c)}(Const)

  \and
  \inferrule{%
    \Gamma \vdash e_1 : \tau_1 \\ \Gamma \vdash e_2 : \tau_2 \\
    \tau_1 \subtype \zero \rightarrow \one \\
    \tau_2 \subtype \dom(\tau_1)
  }{%
    \Gamma \vdash e_1 e_2 : \tau_1 \tau_2
  }
  (App)

  \and
  \inferrule{%
    \forall (\sigma \rightarrow \tau) \in \mathbb{I}, \\
      \Gamma; \ofTypeP{p}{\sigma} \vdash e: \tau' \\
      \tau' \subtype \tau
  }{%
    \lambda^\mathbb{I}p.e : \text{TypeOf}(\mathbb{I})
  }
  (Abs)

  \and
  \inferrule{%
    \Gamma \vdash e : \tau \\
    % TODO: fix display of cases here
    \begin{cases}
      \tau \not\subtype \lnot t \Rightarrow \Gamma \vdash e_1 : \sigma_1 &
      \tau \not\subtype t \Rightarrow \Gamma \vdash e_2 : \sigma_2
    \end{cases}
  }{%
    \Gamma \vdash ((e \in t) ? e_1 : e_2) : \sigma_1 \vee \sigma_2
  }
  (Tcase)

  \and
  \inferrule{%
    \forall i \in \discrete{0}{n},
      \Gamma; x_i : \tau_{(1,i)}; \ldots; x_n: \tau_{(n,i)} \vdash e_i : \sigma_i \\
    \forall (i,j) \in \discrete{0}{n} \times \discrete{1}{n},
      \sigma_j \subtype \tau_{(i,j)}
  }{%
    \Gamma \vdash \text{let } x_1 = e_1; \ldots{}; x_n = e_n \text{ in } e_0 : \sigma_0
  }
  (Let)

  \and
   -- Operators --

   \and
   -- Records --
\end{mathpar}

\end{document}
